

\documentclass[11pt,a4paper,sans]{moderncv}

% moderncv themes
\moderncvstyle{oldstyle}                           % style options are 'casual' (default), 'classic', 'oldstyle' and 'banking'
\moderncvcolor{grey}                               % color options 'blue' (default), 'orange', 'green', 'red', 'purple', 'grey' and 'black'
%\renewcommand{\familydefault}{\sfdefault}         % to set the default font; use '\sfdefault' for the default sans serif font, '\rmdefault' for the default roman one, or any tex font name
%\nopagenumbers{}                                  % uncomment to suppress automatic page numbering for CVs longer than one page

% character encoding
\usepackage[utf8]{inputenc}                       % if you are not using xelatex ou lualatex, replace by the encoding you are using
%\usepackage{CJKutf8}                              % if you need to use CJK to typeset your resume in Chinese, Japanese or Korean

% adjust the page margins
\usepackage[scale=0.75]{geometry}
%\setlength{\hintscolumnwidth}{3cm}                % if you want to change the width of the column with the dates
%\setlength{\makecvtitlenamewidth}{10cm}           % for the 'classic' style, if you want to force the width allocated to your name and avoid line breaks. be careful though, the length is normally calculated to avoid any overlap with your personal info; use this at your own typographical risks...

% personal data
\name{Micah}{Dameron}
\title{EVM Researcher}                               % optional, remove / comment the line if not wanted
\address{18184 Swiss Dr}{Spring Lake, MI}{49456}% optional, remove / comment the line if not wanted; the "postcode city" and and "country" arguments can be omitted or provided empty
\phone[mobile]{+1~(970)~415~7158}                   % optional, remove / comment the line if not wanted
\email{micah.dameron@consensys.net}                               % optional, remove / comment the line if not wanted
\homepage{github.com/chronaeon}                         % optional, remove / comment the line if not wanted
\quote{In the history of computers there has never been an unexpectedly short debugging period. \\ --Steven Levy}                                 % optional, remove / comment the line if not wanted

% to show numerical labels in the bibliography (default is to show no labels); only useful if you make citations in your resume
%\makeatletter
%\renewcommand*{\bibliographyitemlabel}{\@biblabel{\arabic{enumiv}}}
%\makeatother
%\renewcommand*{\bibliographyitemlabel}{[\arabic{enumiv}]}% CONSIDER REPLACING THE ABOVE BY THIS

% bibliography with mutiple entries
%\usepackage{multibib}
%\newcites{book,misc}{{Books},{Others}}
%----------------------------------------------------------------------------------
%            content
%----------------------------------------------------------------------------------
\begin{document}
%-----       letter       ---------------------------------------------------------
% recipient data
\recipient{Diligenc3 Recruitment Team}{ConsenSys\\49 Bogart St\\Brooklyn, NYC, USA}
\date{\today}
\opening{Dear Diligenc3,}
\closing{Yours in Blockchain,}
\enclosure[Attached]{curriculum vit\ae{}}          % use an optional argument to use a string other than "Enclosure", or redefine \enclname
\makelettertitle

I think I might be able to make a good addition to your team as a security engineer. I've been working intimately with computers since my boyhood, but I never felt motivated to get really good at any language until learning about Ethereum. Solidity, once I realized what it was capable of, was the first language that really held my interest. But when starting to learn Solidity I was disappointed to find that the machine it runs on had a faulty specification. I didn't really want to understand Solidity until I could understand the machine because I knew that understanding the machine would be the only way to estimate my own margin of error correctly when it comes to taking security measures for smart contracts. 

I saw that ConsenSys was hiring for summer interns, so I applied and upon being accepted I decided to team up with Alethio, my intent being to work on the \textit{Ethereum Ontology} (EthOn) and learn as much as I could about the EVM so that I could better understand Solidity and what made it tick. I devoted my summer to creating accurate technical descriptions for EthOn and made good progress--improving the Ethereum Ontology's descriptions--and picking up skills with \textit{Git}, \textit{Bash}, and \textit{XML} along the way. 

As the summer drew to a close however, I did not feel my curiosity had been satisfied about understanding Ethereum as a whole system, and its collection of underlying parts. This feeling was confirmed when Johannes, the spoke lead for Alethio, opined that the Ethereum Ontology would not be following the definitions in the Yellowpaper anymore, but would begin to create its own abstractions from onchain data -- different from those found in the Yellowpaper. This seemed arbitrary to me, and I was inwardly disappointed in my ambitions to understand the EVM clearly. 

So I decided to hop into the belly of the beast; the last two weeks of my internship I started notating the Yellowpaper in my own style and, pulling out all the mental stops, made up my mind to understand it from cover to cover, regardless of the fact that the math seemed highly intimidating to me. I also realized my work would be much more constructive for the whole Ethereum ecosystem if I turned it into a project that everyone could read, and so everyone could understand the Yellowpaper the way I wanted to. In addition to scribing technical calls for the \textit{Alignment Circle}, this is primarily what I've been working on (as a contractor) since my internship ended in August. The work is coming along nicely and I believe it is adequately evolving to fit the needs of myself and the community as it goes.\footnote{The majority of my work so far can be found in github.com/chronaeon/Modules.git} 

I want to utilize my increasing knowledge of the minutiae of the EVM to help make smart contracts more secure. I'm a natural at discovering ways to use anything, its intended purposes. I'm also interested in information security more generally and have an insatiable desire to learn everything I can about a field until I've  mastered it. Using the skills I've learned over the summer and throughout the fall, I'd like to spend my time searching through all of the logic in a smart contract, from the abstract user-interface level that the user sees and interacts with, to the concrete machine level that only experienced devs and \textit{Yellowpaper} readers tend to interact with, to identify bugs, oversights, and undiscovered code behaviors that reveal  potential and/or actual security faults in the code. 

\makeletterclosing

\end{document}


%% end of file `template.tex'.
